\documentclass[
  tccRedacao,           % tipo de trabalho: tccProposta, tccRedacao, relatorioEstagio
  12pt,                 % tamanho da fonte (12pt, como recomendado pela ABNT NBR 14724)
  twoside,              % impressão: oneside (apenas no anverso); twoside (anverso e verso)
  a4paper,              % tamanho da folha 
  chapter=TITLE,        % título do capítulo em CAIXA ALTA
  english,              % hifenizar palavras em inglês (no abstract, por exemplo)
  brazil                % hifenizar palavras em português-brasileiro (padrão no abntex2). Deve ser incluída em caso de mais de um idioma.
]{fisufpel-abntex2}

%%%%%%%%%%%%%%%%%%%%%%%%%%%%%%%%%%%%%%%%%%%%%%%%%
%%
%% Macros abntex2 para configuração do documento (autor, titulo, orientação ....)
%%
%
% Autor. Para varios autores use \and entre eles. A impressão é feita pela macro \imprimirautor da abntex2
%
\autor{Nome Sobrenome}           % Nome completo do autor 
\autortrabalho{Sobrenome}{Nome}  % {Sobrenome}{Nome}, usado no resumo, abstract e ficha catalográfica
%
% Titulo do trabalho (língua vernácula). A impressão é feita pela macro \imprimirtitulo da abntex2
%
\titulo{??Título do TCC??}         % título do TCC
\subtitulo{??se houver??}         % subtítulo. Digitar tudo em minúsculo. Deixar em branco se não houver subtítulo
%
% Titulo do trabalho (em inglês).
%
\titleA{??TCC Title??}    % no idioma secundario
\subtitleA{?? if nedeed??}       % subtítulo. Deixar em branco se não houver
% 
% Localidade: cidade da defesa. Colocar sigla do estado para cidades com mesmo nome. A impressão é feita pela macro \imprimirlocal da abntex2
% 
\local{Pelotas}  % cidade 
%
% Data: ano de depósito. A impressão é feita pela macro \imprimirdata da abntex2
%
\data{\the\year}  % ano de defesa, usado na capa. Pode ser passado um ano especifico: \data{2018}
%
% Instituição. A impressão é feita pela macro \imprimirinstituicao da abntex2 
%
\instituicao{Universidade Federal de Pelotas}
%
% Unidade. A impressão é feita pela nova macro \imprimirunidade  
%
\unidade{Instituto de Física e Matemática}
%
% Curso
%
\curso{Curso de Bacharelado em Física}   
%
% Orientador. A impressão é feita pela macro \imprimirorientador. O rótulo [...] pode ser impresso pela macro \imprimirorientadorRotulo
%
\orientador[Orientador]{Nome do orientador}      % em caso de orientadora, mudar rótulo para [Orientadora]
\nomeorientador{Sobrenome}{Nome}                 % para ser usado na ficha catalográfica
\titulacaoorientador{Titulação do orientador}    % em caso de orientadora, mudar para Doutora em ????
%
% Coorientador. A impressão é feita pela macro \imprimircoorientador. O rótulo [...] pode ser impresso pela macro \imprimircoorientadorRotulo 
%
\coorientador[]{}            %  se não tiver coorientador, deixar os dois campos em branco.
\nomecoorientador{}{}        %  use no formato {Sobrenome}{Nome}, para ser usado na ficha catalográfica
%
% Tipo do Trabalho. A impressão é feita pela macro \imprimirtipotrabalho da abntex2
%
\tipotrabalho{Trabalho de Conclusão de Curso}
%
% Codigo cutter para ficha catalográfica (obrigatória para versão final do TCC)
%
\cutter{} % Se não tiver, deixe em branco
%
% código CDD para ficha catalográfica (se não tiver, deixe em branco)
%
\CDD{}
%
%  bibliotecário: CRB e Nome Completo do bibliotecário (para a ficha catalográfica)
%
\bibliotecario{}{}
%
% data de aprovação
%
\datadefesa{\today}  % colocar a data da defesa no formato dia do mês de ano (por exemplo: 12 de junho de 2019)
%
% Membros da Banca (outros, além do Orientador. Utilize o formato \membroA{Titulo}{Nome}{instituição})
%                        para o Título, use a titulação dos membros da banca como declarado nos CVs.
%                        A instituição é a de origem do docente.
%
\membroA{Doutor em ???}{??? membro 1 ???}{Universidade ???}  
\membroB{Doutor em ???}{??? membro 2 ???}{Universidade ???}    
\membroC{Doutor em ???}{??? membro 3 ???}{Universidade ???}     

% Palavras Chaves (Portugues): A-J
\chaveA{Palavra 1}
\chaveB{Palavra 2}
\chaveC{Palavra 3}
%=================================

% Palavras Chaves (Ingles): A-J
\keywordA{Keyword 1}
\keywordB{Keyword 2}
\keywordC{Keyword 3}
%=================================

\ifpdf
\pdfcompresslevel 9  %% nível de compressão (0 -- 9)
\fi

\makeindex  %  compila o indice (não retirar)

%%
%%    INICIO DO DOCUMENTO
%%

\begin{document}

\OnehalfSpacing    % Espaçamento entre linhas igual a um espaço e meio (1.5) -- Não retirar

%%
%%  Elemento externo (não incluído na contagem do número de páginas totais)
%%
\imprimircapa     % macro da abntex2 para CAPA (obrigatória no TCC)

\pretextual 

%%
%% ELEMENTOS PRÉ-TEXTUAIS (são incluídos na contagem de páginas totais)
%%
%%    Comente os elementos opcionais abaixo caso não os deseje.
%%
\imprimirfolhaderosto         % macro da abntex2 para Folha de Rosto (obrigatória) SEM Ficha Catalográfica no verso 
%\imprimirfolhaderosto*       %   usada na versão frente e verso (twoside) do TCC E com a Ficha Catalográfica no verso 

%\imprimirfichacatalografica   % Ficha Catalográfica (obrigatória no TCC, após aprovação)
%\includepdf{ficha.pdf}        % Ficha Catalográfica (obrigatória no TCC, após aprovação para homologar)

\imprimirfolhadeaprovacao     % Folha de Aprovação, para assinar no dia da defesa (obrigatória)

%\includepdf{folhadeaprovacao_digitalizada.pdf}   % Folha de Aprovação digitalizada (obrigatória para versão aprovada) 

\include{./docs/dedicatoria}    % dedicatória (opcional)
\include{./docs/agradecimentos} % agradecimentos (opcional)
\include{./docs/epigrafe}       % epígrafe (opcional)
\include{./docs/resumo}         % resumo (obrigatória apenas no TCC)
\include{./docs/abstract}       % abstract (obrigatória apenas no TCC)
%
% LISTA DE FIGURAS, CÓDIGOS E TABELAS
% ================================================

\listoffigures  % Lista de Figuras (obrigatória apenas no TCC)
%
% não retirar a estrutura abaixo
%
\pdfbookmark[0]{\lstlistlistingname}{lol}
\begin{KeepFromToc}  % comando KeepFromToc é usado para que a "Lista de códigos" não seja inseridas no Sumário. 
\lstlistoflistings   % Lista de códigos fonte (opcional).
                     % Usa pacote listings. O ambiente do código está definido para Fortran 90. 
\end{KeepFromToc}
\cleardoublepage
%

\listoftables  % Lista de Tabelas (opcional)

% ================================================
% LISTA DE SIGLAS E ABREVIATURAS 
% ==============================================================
\include{./docs/abreviaturasesiglas}  % Lista de Abreviaturas e Siglas (opcional)
% ==============================================================
% LISTA DE SÍMBOLOS
% ==============================================================
\include{./docs/simbolos}   % Lista de Símbolos (opcional)
%
% SUMÁRIO
% ==============================================================
\renewcommand\contentsname{Sumário}  % não retirar
\pdfbookmark[0]{\contentsname}{toc}
\tableofcontents*       % Sumário (obrigatório)
\cleardoublepage

% ==============================================================
%%
%%  ELEMENTOS TEXTUAIS 
%%
%% =============================================================================
%
% Hierarquia (ABNT):
%
% \chapter{???}                         --- seção primária   | 1.
%    \section{???}                      --- seção secundária | 1.1
%       \subsection{???}                --- seção terciária  | 1.1.1
%          \subsubsection{???}          --- seção quaternária| 1.1.1.1  
%              \subsubsubsection{???}   --- seção quinária   | 1.1.1.1.1
%
% ==============================================================================

\textual             % Inicia os elementos Textuais (não retirar)

\pagestyle{simple}   % Retira a linha e titulo no verso das páginas (headings é o padrão da classe memoir) (não retirar)

% =============================================================================
%
%  Capítulos da parte textual
%
%            Inclua os arquivos .tex correspondentes (armazenados no diretório docs), a medida que a Proposta for redigida.
%            As figuras, caso armazenadas no diretório figs, devem ter seus caminhos indicados de forma apropriada.
%
% ==============================================================================

\include{./docs/capit01}   %% 1o capítulo, começo do texto e da numeração das páginas
\include{./docs/capit02}   %% 2o capítulo 
\include{./docs/capit03}   %% 3o capítulo 
\include{./docs/capit04}   %% 4o capítulo 
\include{./docs/capit05}   %% 5o capítulo

%%
%%    ELEMENTOS PÓS-TEXTUAIS
%%
%\postextual    % Inicia os elementos pós-textuais. Se habilitada, coloca um espaço no Sumário entre os elementos textuais e pós-textuais
%
% REFERENCIAS (obrigatório)
% 
\bibliography{./docs/bibtex.bib}  % as entradas podem estar em arquivos externos .bib distintos (separados por vírgula)  
\hypertarget{references}{}

%
% Glossário (opcional)
%
%\include{./docs/glossario}   

%
% APÊNDICES (opcional)
%
%   inclua os apêndices através de arquivos .tex (dentro do diretório docs) nos \include abaixo
%

\begin{apendicesenv}   % início dos apêndices

\partapendices  % imprime uma folha contendo a expressão Apêndices centralizado na folha (exigido pela UFPel)

\include{./docs/apendice1} 
\include{./docs/apendice2}
\include{./docs/apendice3}

\end{apendicesenv}  % fim dos apêndices

%
%  Anexos (optativo)
% 
              
\begin{anexosenv} % início dos anexos

\partanexos   % imprime uma folha contendo a expressão Anexos centralizado na folha (exigido pela UFPel)
  
\include{./docs/anexo1}
%\include{./docs/anexo2}

\end{anexosenv} % fim dos anexos

% INDICE
% ==============================================================================
% Para utilizar o indexamento automatico use no corpo do texto:
%   \index{palavra} 
%
%\printindex          % Índice
% ==============================================================================

\end{document}
