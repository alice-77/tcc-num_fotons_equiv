\documentclass[xcolor=dvipsnames]{beamer}

\usepackage[utf8]{inputenc}
\usepackage[portuguese]{babel}
\usepackage[T1]{fontenc}
\usepackage{amsmath}
\usepackage{indentfirst}
\usepackage{amsfonts}
\usepackage{amssymb}
\usepackage{graphicx}
%\usepackage{axodraw2}
\usepackage{subcaption}
\usepackage{appendixnumberbeamer}
\usepackage{multirow}

% Tikz com snakes
\usepackage{tikz}
\usetikzlibrary{snakes}

% Uso de uma fonte especial que eu acho bonite
%\usepackage[math]{iwona}
\usepackage{fourier}

\usetikzlibrary{3d}
\DeclareMathOperator*{\sen}{sen}
\renewcommand{\vec}{\mathbf}

\usetheme{Madrid}

\definecolor{azulUFPel}{rgb}{0.0, 0.25, 0.55}
\usecolortheme[named=azulUFPel]{structure}
\usefonttheme{serif}

% Uns trecos que eu fiz pra ficar mais ao meu gosto estético
\setbeamertemplate{itemize item}[triangle]
\setbeamertemplate{enumerate item}[square]
\setbeamertemplate{navigation symbols}{}
\setbeamertemplate{section in toc}[circle]

\title{Método dos Fótons Equivalentes}
\subtitle{Revisão e Aplicações}
\author[A. A. S. Pacheco]{
	Alfredo Achterberg S. Pacheco\\
	{\footnotesize Orientado por: Prof.  Dr. Werner Krambeck Sauter}\\
	{\scriptsize Defesa da Proposta de Trabalho de Conclusão de Curso}
}
\institute[UFPel]{
	Curso de Bacharelado em Física - Universidade Federal de Pelotas
}
\date[25 de set., 2023]{25 de Setembro, 2023}
\titlegraphic{
	\includegraphics[height=1.5cm]{./logos/logoUFPEL.png}
}
%\logo{\includegraphics[width=0.15\textwidth]{logoUFPel.png}}

\begin{document}

\frame{\titlepage}

\begin{frame}
\frametitle{Estrutura da Apresentação}
\tableofcontents
\end{frame}

\section{Introdução e Contextualização}

\begin{frame}
	\frametitle{Introdução e Contextualização}

	\begin{columns}
	\column{0.5\textwidth}	
		\begin{figure}
			\includegraphics[width=\columnwidth]{./figs/atlan_normal.jpg}
			\caption{Foto do detector ATLAS do LHC. {\scriptsize Créditos:
			[\url{https://home.web.cern.ch/science/experiments/atlas}]}}
		\end{figure}
	
	\column{0.5\textwidth}
		Colisões de partículas constituem o método experimental mais utilizado
		atualmente para o entendimento da estrutura fundamental da matéria e de
		teste para novos modelos físicos.
	\end{columns}

\end{frame}

\begin{frame}
	\frametitle{Introdução e Contextualização}
	Estudos desse tipo de processo tem longa história na física.
	\begin{itemize}
		\item Como exemplo o trabalho de decréscimo de velocidade de partículas
			$\alpha$ e $\beta$ em meios materiais por N. Bohr;
		\item nesse trabalho, o físico propôs que a interação de partículas
			carregadas pode ser entendida pelo fenômeno eletromagnético de
			dispersão (uma analogia);
		\item em 1924, E. Fermi propôs que os campos de uma partícula carregada
			podem ser aproximados como pulsos de onda ou \textit{fluxos de
			fótons virtuais.}
	\end{itemize}
\end{frame}

\begin{frame}
	\frametitle{Introdução e Contextualização}

	\begin{columns}
	\column{0.6\textwidth}

	Disso, E. J. Williams, em 1933, propôs a generalização relativística do que
	seria o método dos fótons equivalentes.
	\begin{itemize}
		\item O método consiste, de forma introdutória, em obter o número de 
			fótons virtuais do campo eletromagnético de uma partícula a
			partir da transformada de Fourier dos mesmos campos;
		\item este consiste de uma aproximação \textit{semi-clássica} para
			o cálculo desses fótons virtuais.
	\end{itemize}

	\column{0.4\textwidth}

	\begin{figure}
		\begin{tikzpicture}[thick]
	\draw(-2,2) ellipse (0.15 and 0.75);
	\draw(0,0) ellipse (0.15 and 0.75);
	\node[left] (z1) at (-2.15,2) {$Z_1e$};
	\node[right] (z2) at (0.15,0) {$Z_2 e$};

	\draw[->](-2,2.75) -- (-2,3.75);
	\draw[->, rotate around={15:(-2,2.75)}](-2,2.75) -- (-2,3.75);
	\draw[->, rotate around={-15:(-2,2.75)}](-2,2.75) -- (-2,3.75)
		node[right]{$\vec{E}$};

	\draw[->](-2,1.25) -- (-2,0.25);
	\draw[->, rotate around={15:(-2,1.25)}](-2,1.25) -- (-2,0.25);
	\draw[->, rotate around={-15:(-2,1.25)}](-2,1.25) -- (-2,0.25);

	\draw[->](0,0.75) -- (0,1.75);
	\draw[->, rotate around={15:(0,0.75)}](0,0.75) -- (0,1.75);
	\draw[thick, ->, rotate around={-15:(0,0.75)}](0,0.75) -- (0,1.75);

	\draw[->](0,-0.75) -- (0,-1.75);
	\draw[->, rotate around={15:(0,-0.75)}](0,-0.75) -- (0,-1.75);
	\draw[->, rotate around={-15:(0,-0.75)}](0,-0.75) -- (0,-1.75);

	\draw[->](-1.85,2) -- (-1,2);
	\draw[->](-0.15,0) -- (-1.15,0) node[below]{$v \approx c$};
\end{tikzpicture}


		\caption{Esquema representando os campos relativísticos de dois íons
		$Z_1$ e $Z_2$}
	\end{figure}
	\end{columns}
\end{frame}

\begin{frame}
	\frametitle{Introdução e Contextualização}

	Desde tais desenvolvimentos, este método aproximativo teve maior aplicação e
	desenvolvimentos na área de interação nuclear e de partículas fundamentais.

	\begin{itemize}
		\item Em especial, focaremos nas colisões ultraperiféricas de íons;
		\item são colisões com maior distância (parâmetro de impacto) e com
			interação dominantemente eletromagnética;
		\item pela interação ser eletromagnética também há menos multiplicidade
			nos estados finais e os resultados experimentais são mais facilmente
			tratados;
		\item fenômenos de interesse nesses processos incluem a produção de
			pares de partículas a partir de colisões de fótons.
	\end{itemize}
\end{frame}

\section{Objetivos do Trabalho}

\begin{frame}
	\frametitle{Objetivos do Trabalho}

	Para a realização do trabalho propomos uma revisão bibliográfica com cálculo
	analítico e computacional de quantidades de interesse dos processos de
	colisão. Para isso, temos os seguintes objetivos específicos:

	\begin{enumerate}
		\item realizar a revisão bibliográfica do método;
		\item realizar o cálculo do fator de forma para o fator de forma
			para diferentes distribuições de carga;
		\item deduzir o número de fótons equivalentes para diferentes
			distribuições de carga;
		\item realizar um estudo mais aprofundado sobre o fenômeno de
			fotoprodução de pares de partícula-antipartícula;
		\item obter as curvas teóricas para as seções de choque de diferentes
			processos de colisão e compará-las com as curvas experimentais.
	\end{enumerate}

\end{frame}

\section{Seção de Choque Diferencial e Total}
\begin{frame}
	\frametitle{Seção de Choque Diferencial e Total}

	O problema de interesse do método é o de colisão de partículas carregadas.
	A quantidade de interesse em colisões é a seção de choque.

	\begin{figure}[h]
		\centering
		\includegraphics[width=0.8\textwidth]{./figs/cross_section.jpeg}
		\label{fig_cross_section}
		\caption{Partícula adentrando a região de espalhamento por uma
		seção de área $d\sigma$ e  sendo espalhada em um ângulo sólido
		$d\Omega$.}
	\end{figure}
\end{frame}

\begin{frame}
	\frametitle{Seção de Choque Diferencial e Total}

	\begin{columns}
	\column{0.5\textwidth}
	Da figura temos as diferenciais,
	\begin{gather}
		d\sigma = |b\, db\, d\phi|,\\
		d\Omega = |\sen \theta \, d\theta \, d\phi|.
	\end{gather}
	A razão entre as duas é,
	\begin{equation}
		\frac{d\sigma}{d\Omega} = \bigg| \frac{b}{\sen \theta}
		\frac{db}{d\theta} \bigg|. \label{diff_cross_section}
	\end{equation}
	Que é a seção de choque diferencial.

	\column{0.5\textwidth}
	A seção de choque total vem pela integral sobre $\Omega$,
	\begin{equation}
		\sigma = \int \frac{d\sigma}{d\Omega} \sen \theta \, d\theta \, d\phi .
	\end{equation}

	\end{columns}

\end{frame}

\begin{frame}
	\frametitle{Seção de Choque Diferencial e Total}
	\begin{block}{Isto para uma partícula incidente individual!}
		Estamos levando em conta uma partícula individual. Se quisermos tratar
		um feixe de partículas, vamos precisar definir a \textit{luminosidade}.
	\end{block}

\end{frame}

\begin{frame}
	\frametitle{Seção de Choque Diferencial e Total}
	\begin{block}{Luminosidade}
		Para um feixe de $N$ partículas com mesma energia atravessando a área
		$d\sigma$, a luminosidade $\mathcal{L}$ é definida como a quantidade
		de partículas que atravessam a região de espalhamento por unidade de
		área por unidade de tempo.
	\end{block}
	Disso, reescrevemos a seção de choque para um feixe de múltiplas partículas,
	\begin{gather}
		dN = \mathcal{L} d\sigma, \\
		\Rightarrow \frac{d\sigma}{d\Omega} = \frac{1}{\mathcal{L}}
		\frac{dN}{d\Omega}.
	\end{gather}
\end{frame}

\section{Demonstração do Método}

\begin{frame}
	\frametitle{Demonstração do Método para Carga Pontual}
	Inicialmente consideramos uma carga em movimento como abaixo.\footnote{A
	partir daqui usaremos unidades naturais ($\hslash = c = 1$).}
	\begin{figure}
	\input{./figs/moving_charge-tikzgraph.tex}
	\caption{Carga $q$ em movimento com velocidade $\vec{v}$ passando por um
		ponto de observação $P$ com parâmetro de impacto $b$ e distância $r$.
		Referencial $\Sigma$ é solidário ao ponto $P$ e $\Sigma '$ é solidário
		à carga pontual $q$.}
	\end{figure}
\end{frame}


\begin{frame}
	\frametitle{Demonstração do Método para Carga Pontual}
	Para o caso com velocidade da partícula em $x_1$, a transformação de
	Lorentz dos campos é,
	\begin{equation}
		\label{eq_field_trans}
		\begin{cases}
		E_1' = E_1 \\
		E_2' = \gamma (E_2 - \beta B_3) \\
		E_3' = \gamma (E_3 + \beta B_2) \\
		\end{cases} \qquad
		\begin{cases}
		B_1 ' = B_1 \\
		B_2 ' = \gamma (B_2 + \beta E_3) \\
		B_3' = \gamma(B_3 - \beta E_2)
		\end{cases},
	\end{equation}
	sendo $\gamma = (1-\beta ^2)^{-1/2}$ e $\beta = v/c$ os parâmetros
	relativísticos da partícula.
\end{frame}


\begin{frame}
	\frametitle{Demonstração do Método para Carga Pontual}
	Escrevendo os campos nas \textit{coordenadas} de $\Sigma$ e depois
	aplicando a transformada de Lorentz temos os campos no \textit{referencial}
	$\Sigma$,
	\begin{gather}
		E_1 (t) = -\frac{q\gamma vt}{(b^2 + \gamma ^2 v^2t^2)^{3/2}}
			\label{eq_field1},\\
		E_2 (t) = \frac{q\gamma b}{(b^2 + \gamma ^2 v^2 t
			^2)^{3/2}}\label{eq_field2},\\ 
		B_3 (t) = \beta E_2(t) \label{eq_field3}.
	\end{gather}
	\begin{block}{Aproximamos estes campos como pulsos de onda.}
		Analisando esses campos podemos notar que $E_2$ e $B_3$ formam um pulso
		de onda na direção $x_1$. Ainda assim, a interação do campo $E_1$ pode
		ser analisada como um pulso de onda pela inserção de um campo magnético
		artificial como aproximação.
	\end{block}
\end{frame}

\begin{frame}
	\frametitle{Demonstração do Método para Carga Pontual}
	\begin{figure}
		\begin{subfigure}[b]{0.4\textwidth}
			\centering
			\begin{tikzpicture}[thick]
	\draw[-] (0,0) --  (3,0)node[right]{$x_1$};
	\draw[-] (0,0) -- (0,2.5) node[above]{$x_2$};
	\draw[-] (0,0) -- (0,0,1) node[below left]{$x_3$};

	\draw[->, blue] (2,1.5) -- (2,3.5) node[left]{$E_2$};
	\draw[->, blue] (2,1.5) -- (3.5,1.5) node[right]{$E_1$};
	\draw[->, blue] (2,1.5) -- (2,1.5,2) node[below]{$B_3$};

	\filldraw (2,1.5) circle(3pt) node[above left] {$P$};
\end{tikzpicture}

			\caption{Campos observados no referencial do ponto $P$.}
		\end{subfigure}
		\hspace{1cm}
		\begin{subfigure}[b]{0.4\textwidth}
			\centering
			\begin{tikzpicture}[thick]
	\draw[-](-1,0) -- (1,0) node[right]{$x_1$};
	\draw[-](0,0) -- (0,2.5) node[above]{$x_2$};

	\filldraw (0,1.5) circle(3pt) node[above right]{$P$};

	\draw[
		->,
		snake = coil,
		segment aspect=0,
		segment length=5pt,
		segment amplitude=5pt,
		line after snake=2pt,
		blue
	] (0,0) -- (0,1) node[left]{$P_2$};
	\draw[
		->,
		snake = coil,
		segment aspect=0,
		segment length=5pt,
		segment amplitude=5pt,
		line after snake=2pt,
		blue
	] (-1.5,1.5) -- (-0.5,1.5) node[above]{$P_1$};
\end{tikzpicture}

			\caption{Pulsos aproximados $P_1$ e $P_2$ atingindo $P$.}
		\end{subfigure}
		\caption{Aproximação chave do método dos fótons virtuais é a de
		substituir os campos elétrico e magnético por pulsos de radiação
		equivalentes.}
	\end{figure}
\end{frame}

\begin{frame}
	\frametitle{Demonstração do Método para Carga Pontual}
	Com isso, iremos calcular agora os espectros de frequência\footnote{A
	energia por unidade de frequência e área de um pulso}, para ambos os
	pulsos. Estes o são
	\begin{gather}
		I_1(\omega , b) = \frac{1}{2\pi} |E_2 (\omega) |^2 ,\\
		I_2 (\omega , b) = \frac{1}{2\pi} |E_1 (\omega)|^2,
	\end{gather}
	em que 
	\begin{equation}
		E_{1,2} (\omega) = \frac{1}{\sqrt{2\pi}} \int _{-\infty}^{+\infty}
		dt \; E _{1,2} (t) e^{i\omega t},
	\end{equation}
	é a transformada de Fourier da parte elétrica dos pulsos.
\end{frame}

\begin{frame}
	\frametitle{Demonstração do Método para Carga Pontual}
	O cálculo da integral para os dois campos leva ao seguinte resultado,
	\begin{gather}
		I_1 (\omega , b) = \frac{1}{\pi ^2} \frac{q^2}{ \beta ^2 b^2}  
		\xi ^2 K_1 ^2 \left( \xi \right), \\
		I_2 (\omega , b) = \frac{1}{\pi ^2} \frac{q^2}{\beta ^2 b^2 }
		\frac{1}{\gamma ^2} \xi^2 K_0 ^2 \left( \xi \right).
	\end{gather}
	onde $\displaystyle \xi \equiv \frac {\omega b}{\gamma v}$ e as
	funções$K_1$ e $K_0$ são as funções modificadas de Bessel.
\end{frame}

\begin{frame}
	\frametitle{Demonstração do Método para Carga Pontual}
	A partir disso, o número de fótons equivalentes pode ser obtido pelo
	espectro de frequência como,
	\begin{equation}
		\begin{split}
		N(\omega , b) &= \frac{1}{\omega} \left[ I_1 (\omega , b) + I_2(\omega
			, b) \right]\\
			&= \frac{1}{\pi ^2} \frac{q^2} {\beta ^2 b^2}
			\frac{1}{\omega ^2} \xi ^2 \left[K_1^2 (\xi ) + \frac{1}{\gamma ^2}
			K_0 ^2 (\xi ) \right]. \label{eq_EP-SPEC}
		\end{split}
	\end{equation}
\end{frame}

\begin{frame}
	\frametitle{Demonstração do Método para Carga Pontual}
	O número de fótons total é dado pela integral de $N(\omega , b)$ sobre os
	parâmetros de impacto,
	\begin{equation}
	\begin{split}
		n(\omega) &= \int _{b_{\text{min}}}^\infty db\; bN(\omega , b) \\ 
		&= \frac{1}{\pi} \frac{2q^2}{\beta ^2} \frac{1}{\omega} \left\{ \xi
		_\text{min} K_0 \left( \xi \right) K_1 \left( \xi _\text{min} \right) -
		\frac{\beta ^2}{2} \xi _\text{min} ^2 \left[ K_1 ^2 \left( \xi
		_\text{min} \right) - K_0 ^2 \left( \xi _\text{min} \right) \right]
		\right\}, \label{eq_EPT}
	\end{split}
	\end{equation}
\end{frame}

\section{Sobre o Fator de Forma}
\begin{frame}
	\frametitle{Sobre o Fator de Forma}
	Para o caso da partícula incidente não ser pontual é introduzido o
	\textit{fator de forma} $F(|\vec{q}|)$. Assim, o $N(\omega , b)$ fica
	escrito como,
	\begin{equation}
		N(\omega , b) = \frac{1}{\pi ^2} \frac{Z^2 \alpha}{\beta ^2 \omega b^2}
		\Bigg| \int du \; u^2 J_1 (u) \frac{F[(u^2 + \xi ^2)/b^2]}{u^2 + \xi ^2}
		\Bigg|^2. \label{eq_EP-SPEC-F}
	\end{equation}
	O fator de forma $F(|\vec{q}|)$ é a transformada de Fourier da distribuição
	de carga $f(\vec{r})$.
\end{frame}

\begin{frame}
	\frametitle{Sobre o Fator de Forma}
	\begin{itemize}
		\item A maior parte das distribuições de carga são esfericamente
			simétricas.
	\end{itemize}
	\begin{table}
		\begin{tabular}{|c|c|}
			\hline 
			$f(r)$ & $F(|\vec{q}|)$ \\
			\hline
			 $\delta (r) / 4 \pi$ & 1 \\
			$\displaystyle \frac{a^3}{8\pi} e^{-ar}$ &
			$\displaystyle \left(\frac{1 + |\vec{q}|^2}{a^2}\right) ^{-2}$\\
			$ \left( a^2 / 2\pi \right)^{3/2} e
			^{-a^2 r^2 / 2}$ & $e^{|\vec{q}|^2 / 2a^2 }$ \\
			 $\displaystyle \begin{cases}
				3/4\pi R^3, & r \leq R \\
				0, & r > R
			\end{cases}$ & $ \displaystyle \frac{3(\sen \alpha - \alpha \cos
			\alpha)}{\alpha}$, $\alpha = |\vec{q}| R$ 
			\\
			\hline
		\end{tabular}
		\caption{Fatores de forma disponíveis para diferentes distribuições de
		carga esfericamente simétricas.}
	\end{table}
\end{frame}

\appendix

\begin{frame}
	\frametitle{Dedução da Transformada de Lorentz para os Campos}
	Sendo os campos elétrico e magnético escritos em termos dos potenciais,
	\begin{gather}
		\vec{E} = - \nabla \Phi - \frac{\partial \vec{A}}{\partial t}, \\
		\vec{B} = \nabla \times \vec{A},
	\end{gather}
	estes são escritos em forma explicitamente covariante usando o tensor
	eletromagnético,
	\begin{equation}
		F^{\mu \nu} = \begin{pmatrix}
			0 & -E_1 & -E_2 & -E_3 \\
			E_1 & 0 & -B_3 & B_2 \\
			E_2 & B_3 & 0 & -B_1 \\
			E_3 & -B_2 & B_1 & 0
		\end{pmatrix}.
	\end{equation}
\end{frame}

\begin{frame}
	\frametitle{Como escrevemos os Campos nas Coordenadas de $\Sigma$}
	\begin{columns}

	\column{0.5\textwidth}
		Os campos como percebidos em $P$, no referencial $\Sigma '$ tem a forma
		\begin{equation}
			E_1 ' = -\frac{qvt'}{{r'} ^3}, \qquad E_2 ' = \frac{qb}{{r'}^3}.
		\end{equation}
		Escrevemos nas \textit{coordenadas} de $\Sigma$ usando,
		\begin{gather}
			t' = \gamma t, \\
			\begin{split}
				r ' &= \sqrt{b^2 + (v t')^2} \\
					&= \sqrt{b^2 + v^2 \gamma ^2 t^2}.
			\end{split}
		\end{gather}

	\column{0.5\textwidth}
		Assim
		\begin{gather}
			E _1 '= - \frac{q\gamma vt}{(b^2 + \gamma ^2 v^2 t^2)^{3/2}},\\
			E _2 '= \frac{qb}{(b^2 + \gamma ^2 v^2 t^2)^{3/2}},
		\end{gather}
		para os quais devemos aplicar a transformação de Lorentz.
	\end{columns}
\end{frame}


\end{document}
