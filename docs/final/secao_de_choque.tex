\chapter{Fenômenos de Colisão de Partículas}
Nesse capítulo, pretendemos fazer uma introdução aos processos de colisão de
partículas e de espalhamento. Colisões de partículas consistem de um dos
métodos experimentais mais utilizados para o estudo das características da
matéria e de seus componentes. Por isso, a motivação para a melhor compreensão
desses processos é evidente.

De forma geral, os experimentos de espalhamento constituem de um bombardeamento
de um alvo por um feixe de partículas com energia bem definida. Alvos podem ser
sólidos, líquidos, gases ou outros feixes de partículas. Disso, espalhamentos
observados nesses experimentos são classificados em \textit{elásticos} e
\textit{inelásticos} a depender da excitação ou não das partículas alvo
\cite{povh6ed}. De forma resumida, no espalhamento elástico as partículas
permanecem as mesmas e não observamos decaimento ou excitação dos níveis
internos de energia. A única transferência de energia nesse caso é na forma de
energia cinética. Para uma partícula incidente $a$ atingindo uma partícula alvo
$b$ o espalhamento elástico assume a forma,
\begin{equation}
	a + b \rightarrow a' + b' .
\end{equation}

