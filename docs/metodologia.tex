\chapter{Metodologia}
Na execução desse projeto de pesquisa pretendemos realizar um leitura mais
aprofundada do método de fótons equivalentes, realizando os cálculos analíticos
e revisando os métodos teóricos de análise dos problemas. Nesse sentido, uma
revisão do método incluirá um cálculo do fluxo de fótons equivalentes para
distribuições pontuais e extensas de carga. Também, uma revisão teórica das
interações eletromagnéticas em colisões relativísticas de íons será feita, com
o intuito de obter um entendimento maior dos fluxos de fótons gerados pelos
íons a velocidades relativísticas.

Para os cálculos que não puderem ser realizados analiticamente, usaremos de
métodos computacionais para obtenção das curvas teóricas. Com isso, faremos uso
da biblioteca GSL (\textit{GNU Scientific Library}) para C e C++, que possui as
funções especiais de interesse, em conjunto com rotinas de integração que serão
úteis no cálculo de integrais como na equação (\ref{eq_EP-SPEC-F})
\cite{gsl_manual}.

Após a obtenção destes, vamos calcular as seções de choque de processos de
fotoprodução para diferentes íons.  Inicialmente processos de colisão $e^-
e^+$, $e^- e^-$, que possuem distribuição de carga pontual. Deduzir a seção de
choque de fotoprodução de pares nessas colisões analiticamente ou obtê-las
numericamente se for o caso. Disso, iremos calcular as seções de choque para
alguns íons da Tabela \ref{tab_ion-par}, nos atentando para os parâmetros
experimentais de energia e luminosidade, e compararemos tais curvas teóricas
com os dados experimentais das respectivas colisões.


