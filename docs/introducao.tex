\chapter{Introdução}

Uma das questões que desperta mais interesse na Física é compreender a
composição da matéria em sua forma mais fundamental.  Almeja-se a obtenção do
conhecimento das características básicas dos componentes do mundo em que
vivemos, as ditas partículas fundamentais e a forma como esses componentes
interagem para criar a matéria da qual o universo é composto. Para tanto, o
método de investigação experimental mais utilizado tanto no passado quanto
atualmente é o de colisão de partículas \cite{griffiths_particle}. Este
consiste em analisar características das partículas fundamentais e dos núcleos
atômicos a partir da forma como estes são espalhados (desviados) quando
lançados uns contra os outros \cite{thomson_particle}.  Características que são
obtidas por meio desse método incluem a carga elétrica desses íons e a
distribuição de carga para partículas extensas, como os núcleos atômicos. Além
disso, é possível investigar as próprias forças fundamentais da natureza nessas
colisões e reações \cite{griffiths_particle}.

Em especial, consideramos o estudo, realizado por Niels Bohr da interação de
partículas aceleradas com a matéria \cite{bohr1913}. O fenômeno tratado foi do
decréscimo de velocidade de partículas $\alpha$ e $\beta$ em meios materiais,
para o qual o autor elaborou que os elétrons dos átomos oscilariam quando
perturbados por forças externas. Nisso, Bohr propôs uma analogia entre o
fenômeno eletromagnético de dispersão e o da interação das partículas
carregadas.

Posteriormente, se propôs que os campos de uma partícula carregada podem ser
aproximados como pulsos de onda e, desta forma, como um fluxo de fótons
virtuais \cite{Fermi1924}. Fermi os usou para tratar dos fenômenos de colisões
de partículas carregadas com núcleos atômicos e para o cálculo da seção de
choque destas colisões. Uma década após a publicação deste trabalho, foi
proposto uma generalização para partículas relativísticas \cite{williams1933}.
Este método, denominado de método dos fótons equivalentes, ainda é utilizado e
desenvolvido, com a generalização na teoria de eletrodinâmica quântica, para
estudos com colisões envolvendo partículas carregadas \cite{BALTZ20081}.

Desde esses trabalhos, o avanço nas áreas de interação nuclear tem aumentado
consideravelmente, uma vez que colisões com interação eletromagnética
permanecem como uma das formas mais convenientes de estudar a estrutura dos
núcleos e núcleons atômicos \cite{harland-lang2023}. Em especial, salientamos
as colisões ultraperiféricas de íons, que trazem consigo formas de estudar a
distribuição de carga e glúons dessas partículas \cite{bertulani2005}. Esse
método permite um estudo de fenômenos interessantes na física de partículas,
como a produção de matéria a partir de radiação eletromagnética, o fenômeno de
fotoprodução de partículas. Nesse processo os fótons colidem gerando estados
finais com massa e carga, em geral, constituídos de pares de partícula e
antipartícula, como elétrons e pósitrons \cite{Gerhard_Baur_1998}.

Com isso posto, essas colisões são observadas em experimentos de colisão de
última geração, como os disponíveis no LHC\footnote{LHC - \textit{Large Hadron
Colider} ou Grande Colisor de Hádrons é um laboratório de colisão de partículas
inaugurado em 2008 sob gerenciamento da Organização Européia de Pesquisas
Nucleares.} \cite{BALTZ20081}. Nestes experimentos, as colisões
ultraperiféricas permitem a obtenção de dados mais limpos (menos partículas
sendo produzidas por colisão e menos ruído detectado nos sensores) do que as
colisões que envolvem interação de força forte, uma vez que a multiplicidade
dos estados finais é menor. Assim, a obtenção das curvas de seção de choque
experimentais e sua comparação com a teoria é mais facilmente realizada.

A partir do que foi exposto aqui, este trabalho propõe uma revisão
bibliográfica do método dos fótons equivalentes, com análise analítica e
computacional dos processos que podem ser estudados a partir do método. Com
isso, pretendemos, não somente realizar um estudo analítico dos campos de uma
carga pontual se movendo, mas também, realizar o cálculo do número de fótons
equivalentes para as diferentes distribuições de carga nuclear propostas na
literatura. Realizar a revisão desses cálculos quando disponíveis e propor
métodos computacionais para o cálculo destes, quando não forem analiticamente
solúveis. Assim, planejamos também obter as seções de choque para alguns íons
que são utilizados em experimentos de colisões e comparar com os dados
experimentais dos experimentos respectivos.




