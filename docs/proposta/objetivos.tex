\chapter{Objetivos}

\section{Objetivos Gerais}
Este trabalho tem como objetivo uma revisão do método de fótons equivalentes,
da sua obtenção e desenvolvimento com aplicações, em especial para as ditas
colisões ultraperiféricas de íons. Procuraremos realizar, além da
revisão do método, aplicações com cálculos do número de fótons equivalentes
para diferentes fenômenos de colisões. Por fim, buscaremos enfatizar a sua
utilidade e relevância em áreas de fronteira na física de partículas.


\section{Objetivos Específicos}
A fim de realizar esses objetivos iremos executar os seguintes objetivos
específicos:
\begin{enumerate}
\item realizar a revisão bibliográfica do método;
\item realizar os cálculos para os fatores de forma para diferentes distribuições
	de carga;
\item deduzir detalhadamente os números de fótons equivalentes para os casos de
	carga pontual e distribuição de carga para diferentes íons, os quais exigem
	uma diferente escolha do parâmetro de impacto;
\item realizar um estudo dos processos de fotoprodução de pares de
	partícula-antipartícula; 
\item obter as curvas teóricas do espectro de frequência, número de fótons
equivalentes e seção de choque para os diferentes processos de colisão;
\item comparar as curvas com os dados experimentais, com interesse nas
colisões ultraperiféricas, onde é observável o fenômeno de fotoprodução.
\end{enumerate}

