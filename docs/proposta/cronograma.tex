\chapter{Cronograma e Plano de Trabalho}
\section{Cronograma}
A fim de cumprir com os objetivos postos, planejamos seguir com o seguinte
cronograma para realização do projeto.

\begin{table}[h]
	\IBGEtab{
		\caption{Cronograma de trabalho a ser realizado.}
	}{
		\begin{tabular}{lccccc}
	\toprule
	& Mês 1 & Mês 2 & Mês 3 & Mês 4 & Mês 5\\
	\midrule \midrule
	1) Revisão bibliográfica & $\times$ & $\times$ & $\times$ & $\times$ &
	\\
	2) Dedução dos fatores de forma & $\times$
		 & $\times$ & $\times$ & & \\
	3) Cálculo do número de fótons equivalentes & & $\times$ & $\times$ &
		$\times$ & \\
	4) Obtenção das seções de choque dos processos & & $\times$ & $\times$
		& $\times$ & \\
	5) Redação do TCC & & & $\times$ & $\times$ & $\times$ \\
	6) Defesa do TCC & & & & & $\times$ \\
	\bottomrule
\end{tabular}

	}{
		\fonte{O autor (2023).}
	}

\end{table}

\section{Plano de Trabalho}

\begin{enumerate}
	\item \textbf{Revisão bibliográfica:} Procuraremos aprofundar o referencial
	teórico do trabalho. Em especial, para obter mais entendimento das
	colisões ultraperiféricas e dos processos eletromagnéticos dessas
	colisões;
	\item \textbf{Dedução dos fatores de forma:} Obter os fatores de forma
	analiticamente ou numericamente para diferentes distribuições de carga
	disponíveis na literatura;
	\item \textbf{Cálculo do número de fótons equivalentes:} Calcular os
	$N(\omega , b)$ e $n(\omega)$ para as diferentes distribuições de
	carga, utilizando os fatores de forma já calculados;
	\item \textbf{Obtenção das seções de choque:} Obter as seções de choque
	para os diferentes íons usando distribuições de carga recomendadas na
	literatura para cada um, em conjunto com os fatores de forma e números
	de fótons equivalentes calculados;
	\item \textbf{Redação do TCC:} Com o que foi obtido, tanto na revisão
	bibliográfica quanto no estudo realizado dos processos, escrever uma
	revisão do que foi obtido, em conjunto com os resultados;
	\item \textbf{Defesa do TCC:} Após isso, defender o trabalho final.
\end{enumerate}

